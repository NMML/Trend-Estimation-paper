\documentclass[12pt,letter]{report}
%\usepackage[utf8]{inputenc}

%% fonts
\usepackage[charter]{mathdesign}
\usepackage[scaled=.95]{inconsolata}

%% page margins, inter-paragraph space and no chapters
\usepackage[margin=1in]{geometry}
\setlength{\parskip}{0.5em}

\usepackage{lineno, natbib}

%%% for memisc
%\usepackage{booktabs}
%\usepackage{dcolumn} 

%%% define a dark blue
%\usepackage{color}
%\definecolor{darkblue}{rgb}{0,0,.5} 
%
%%% hyperlinks to references
%\usepackage{hyperref}
%\hypersetup{colorlinks=true, linkcolor=darkblue, citecolor=darkblue, filecolor=darkblue, urlcolor=darkblue}

\begin{document}
\begin{center}
\Large A GENERAL FRAMEWORK FOR ESTIMATING ABUNDANCE TRENDS FROM ECOLOGICAL MONITORING DATA
\bigskip\\
\normalsize
{\sc Jay M. Ver Hoef\footnotemark[1], Devin S. Johnson, Lowell Fritz, other PEP-ers?}\smallskip\\
{\em National Marine Mammal Laboratory, NOAA\\
7600 Sand Point Way NE, Seattle,
WA 98115 USA }\\ \medskip
\end{center}
\footnotetext[1]{Email: jay.verhoef@noaa.gov}

\raggedright \setlength{\parindent}{0.3in}
\renewcommand{\baselinestretch}{1.7}\normalsize
\clubpenalty=0
 \linenumbers

{\em Abstract.\ } xxx.

{\em Key words: xxx}

\centerline{\sc Introduction}

Estimating ``trends'' from survey data has a long standing tradition in ecology and resource management. Estimates of trends are often used to inform resource management decisions, invoke resource protection measures, or remove resource protection measures after satisfactory growth in abundance of a population has been reached. The term, ``trend,'' is a nebulous concept, however, with changing definition depending on the situation at hand. In the broadest sense, a trend, is simply the change in abundance over a specified time interval, of which both components may change depending on the question being asked of the data.   

...

The work presented here is motivated by on-going monitoring of two marine mammal species in Alaska, harbor seals ({\it Phoca vitulina}) and Steller sea lions ({\it Eumetopias jubatus}).

...
   
\centerline{\sc Methods}

\centerline{\sc Case study: Harbor seal survey}

\centerline{\sc Case study: Steller sea lion survey}

\centerline{\sc Discussion}

\end{document}
